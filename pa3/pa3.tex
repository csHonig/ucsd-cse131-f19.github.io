
\documentclass[10pt, oneside]{article}
\usepackage[letterpaper, scale=0.9, centering]{geometry}
\usepackage{graphicx}			
\usepackage{url,hyperref}
\usepackage{amssymb,amsmath,mathpartir}
\usepackage{currfile,xstring,multicol,array}
\usepackage[dvipsnames]{xcolor}
\usepackage[labelformat=empty]{caption}

\title{UCSD CSE131 F19 -- Copperhead}

\usepackage{listings}
\pagestyle{empty}
\setlength{\parindent}{0em}
\begin{document}
\maketitle 

{\bf Due Date:} 11pm Wednesday, October 23 \hspace{2em} {\bf Open to Collaboration}

You will implement Copperhead, a language like Boa extended with a static type
system, variables, and while loops.

Classroom: \url{https://classroom.github.com/a/y0UCOlDY} \hspace{1em} Github: \url{http://github.com/ucsd-cse131-f19/pa3-student}


\subsection*{Syntax}

The concrete syntax and type language for Copperhead is below. We use $\cdots$ to
indicate \textit{one or more} of the previous element. So a while expression has at
least two sub-expressions, and a let expression at least one binding and body
expression.

\[
\begin{array}{ll}
\begin{array}{lrl}
e & := & n \mid \texttt{true} \mid \texttt{false} \\
  & \mid  & \texttt{(let (($x\ e$) $\cdots$) $e \cdots$)} \\
  & \mid  & \texttt{(if $e\ e \ e$)} \\
  & \mid  & \texttt{($op_2$ $e\ e$)} \mid \texttt{($op_1$ $e$)} \\
  & \mid  & \texttt{(while $e\ e \cdots$)} \mid \texttt{(set $x\ e$)} \\
op_1 & := & \texttt{add1} \mid \texttt{sub1} \mid \texttt{isNum} \mid \texttt{isBool} \\
op_2 & := & \texttt{+} \mid \texttt{-} \mid \texttt{*} \mid \texttt{<} \mid \texttt{>} \mid \texttt{==} \\
n & := & \textrm{63-bit signed number literals} \\
x & := & \textrm{variable names} \\
\end{array}
&
\begin{array}{lrl}
\tau & := & \textsf{Num} \mid \textsf{Bool} \\
\Gamma & := & \{x:\tau,\cdots\} \\
\Gamma[x] & \textit{means} & \textit{look up the type of } x \textit{ in } \Gamma \\
(x,\tau)::\Gamma & \textit{means} & \textit{add } x \textit{ to } \Gamma \textit{ with type } \tau \\
\Gamma \vdash e : \tau & \textit{means} & \textit{in environment } \Gamma, e \textit{ has type } \tau \\
\end{array}
\end{array}
\]

\subsection*{Semantics}

The behavior of the existing forms is largely same as in Boa, with a few
modifications:

\begin{itemize}
\item \texttt{let} expressions in Copperhead can have multiple body
expressions. These expressions should be evaluated in order, and the result
of the let expression is the result of the last expression in the body.
\item The {\bf static type checker} for Copperhead eliminates the need to check
tags of operands to binary operators like \texttt{+}. Overflow is
checked and reported at runtime as in Boa.
\item The built-in variable \texttt{input} must always be a number. If a user
gives a non-number value, the runtime should report a dynamic error that
includes the string \texttt{"input must be a number"}.
\end{itemize}

The existing static errors (duplicate bindings and unbound identifiers)
should be reported as in Boa.

There are two new expressions in Copperhead, as well:

\begin{itemize}

\item \texttt{set} expressions update the value of a variable. The behavior
of a \texttt{set} expression is to evaluate its subexpression, then set the
value of the named variable to the result of that subexpression. The result
of the entire \texttt{set} expression is the new value, and its effect is to
make future accesses of that variable get the updated value.

\item \texttt{while} expressions evaluate a condition and body repeatedly.
The condition expression is the first one that appears in the while
expression -- it should evaluate to a boolean (the type rules below enforce
this), and if it evaluates to \texttt{true} the body expressions evaluate in
order. This process is repeated until the condition evaluates to
\texttt{false}, the body is not executed, and the entire \texttt{while}
expression evaluates to \texttt{false}.

\end{itemize}

\subsection*{Type Checking}

Copperhead has the typing rules shown in figure 1. If an expression cannot be
typed according to these rules, the compiler should report a static error
containing \texttt{"Type mismatch"}. A few rules benefit from some extra
explanation:

\begin{itemize}

\item TR-Sequence describes type-checking a sequence of expressions, as found
in the body of while and let expressions. To type-check a sequence, all the
expressions must type to \emph{some} type, though they can be different
across expressions. The type of the entire sequence is the type of the
\emph{last} expression, $\tau_n$.

\item TR-LetBindings describes type-checking of let expressions with multiple
bindings in terms of one binding at a time. Since the first binding is
visible in the second, the second in the third, and so on, this rule proceeds
one binding at a time. There is a separate rule, TR-Let-One, to handle the
case of a single binding.\footnote{You may find the implied binding-at-a-time
matching useful as a suggestion of an implementation strategy.}



\end{itemize}

\begin{figure}
\begin{mathpar}
\inferrule*[Left=TR-Num]{}{\qquad n : \textsf{Num}}
\and \and
\inferrule*[Left=TR-True]{}{\qquad \texttt{true} : \textsf{Bool}}
\and \and
\inferrule*[Left=TR-False]{}{\qquad \texttt{true} : \textsf{Bool}}
\\
\inferrule*[Left=TR-Plus]
{\Gamma \vdash e_1 : \textsf{Num} \and \Gamma \vdash e_2 : \textsf{Num}}
{\Gamma \vdash \texttt{(+ $e_1\ e_2$)} : \textsf{Num}}
\and
\inferrule*[Left=TR-Minus]
{\Gamma \vdash e_1 : \textsf{Num} \and \Gamma \vdash e_2 : \textsf{Num}}
{\Gamma \vdash \texttt{(- $e_1\ e_2$)} : \textsf{Num}}
\\
\inferrule*[Left=TR-Times]
{\Gamma \vdash e_1 : \textsf{Num} \and \Gamma \vdash e_2 : \textsf{Num}}
{\Gamma \vdash \texttt{(* $e_1\ e_2$)} : \textsf{Num}}
\\
\inferrule*[Left=TR-Add1]
{\Gamma \vdash e : \textsf{Num}}
{\Gamma \vdash \texttt{(add1 $e$)} : \textsf{Num}}
\and
\inferrule*[Left=TR-Sub1]
{\Gamma \vdash e : \textsf{Num}}
{\Gamma \vdash \texttt{(sub1 $e$)} : \textsf{Num}}
\\
\inferrule*[Left=TR-IsBool]
{\Gamma \vdash e : \tau}
{\Gamma \vdash \texttt{(isBool $e$)} : \textsf{Bool}}
\and
\inferrule*[Left=TR-IsNum]
{\Gamma \vdash e : \tau}
{\Gamma \vdash \texttt{(isNum $e$)} : \textsf{Bool}}
\\
\inferrule*[Left=TR-Less]
{\Gamma \vdash e_1 : \textsf{Num} \and \Gamma \vdash e_2 : \textsf{Num}}
{\Gamma \vdash \texttt{(< $e_1\ e_2$)} : \textsf{Bool}}
\and
\inferrule*[Left=TR-Greater]
{\Gamma \vdash e_1 : \textsf{Num} \and \Gamma \vdash e_2 : \textsf{Num}}
{\Gamma \vdash \texttt{(> $e_1\ e_2$)} : \textsf{Bool}}
\\
\inferrule*[Left=TR-Equals]
{\Gamma \vdash e_1 : \tau_1 \and \Gamma \vdash e_2 : \tau_2}
{\Gamma \vdash \texttt{(== $e_1\ e_2$)} : \textsf{Bool}}
\\
\inferrule*[Left=TR-Id]{\Gamma[x] = \tau}{\Gamma \vdash x : \tau}
\and
\inferrule*[Left=TR-Let-One]
{\Gamma \vdash e_1 : \tau_1 \and (x,\tau_1)::\Gamma \vdash e_b \cdots : \tau}
{\Gamma \vdash \texttt{(let (($x\ e_1$)) $e_b \cdots$)} : \tau}
\\
\inferrule*[Left=TR-Let-Bindings]
{\Gamma \vdash e_1 : \tau_1 \and (x_1,\tau)::\Gamma \vdash \texttt{(let (($x_2\ e_2$) $\cdots$) $e_b$)} : \tau}
{\Gamma \vdash \texttt{(let (($x_1\ e_1$) ($x_2$ $e_2$) $\cdots$) $e_b \cdots$)} : \tau}
\\
\inferrule*[Left=TR-Sequence]
{\Gamma \vdash e_1 : \tau_1 \cdots \Gamma \vdash e_n : \tau_n}
{\Gamma \vdash e_1 \cdots e_n : \tau_n}
\and
\inferrule*[Left=TR-Set]
{\Gamma \vdash e : \tau \and \Gamma[x] = \tau}
{\Gamma \vdash \texttt{(set $x\ e$)}: \tau}
\\
\inferrule*[Left=TR-If]{\Gamma \vdash e_1 : \textsf{Bool} \\ \Gamma \vdash e_2 : \tau \\ \Gamma \vdash e_3 : \tau}{\Gamma \vdash \texttt{(if $e_1\ e_2\ e_3$)} : \tau}
\and
\inferrule*[Left=TR-While]
{\Gamma \vdash e_1 : \textsf{Bool} \\ \Gamma \vdash e_2 \cdots : \tau}
{\Gamma \vdash \texttt{(while $e_1\ e_2 \cdots$)} : \textsf{Bool}}
\end{mathpar}
\caption{Figure 1: Typing rules for Copperhead}
\end{figure}


\subsection*{Extensions}

These are optional and not for credit, but are interesting to try and discuss
in office hours or with your peers:

\begin{enumerate}

\item Modify the compiler to use type information in code generation to
dramatically simplify the compilation of \texttt{isNum} and \texttt{isBool}.

\item Add a new type to the definition of \texttt{typ} called
\texttt{NumOrBool}, and enable booleans as \texttt{input}. Write a type rule
for each of the following cases that exploits the behavior of \texttt{isNum}
and \texttt{isBool} and your knowledge of control flow to make them
type-check safely:

\texttt{(if (isNum input) (< input 1) input)}

\texttt{(if (isBool input) (if input 10 5) (+ input 10))}

Check that your solution works in general for any identifier that has type
\texttt{NumOrBool}. Implement a less restrictive version of \texttt{if} that
allows differing types across the then and else branches.

\end{enumerate}



\end{document}