\documentclass[10pt, oneside]{article}
\usepackage[letterpaper, scale=0.9, centering]{geometry}
\usepackage{graphicx}			
\usepackage{url,hyperref}
\usepackage{amssymb,amsmath,mathpartir}
\usepackage{currfile,xstring,multicol,array}
\usepackage[dvipsnames]{xcolor}
\usepackage[labelformat=empty]{caption}

\title{UCSD CSE131 F19 -- PA5}

\usepackage{listings}

\lstset{
  basicstyle=\ttfamily,
  mathescape,
  columns=fullflexible
}

\setlength{\parindent}{0em}
\setlength{\parskip}{0.5em}
\begin{document}
\maketitle 

{\bf Due Date:} 11pm Wednesday, November 13 \hspace{2em} {\bf Open to Collaboration}

Classroom: \url{https://classroom.github.com/a/z5lJNg9F} \hspace{1em} Github: \url{https://github.com/ucsd-cse131-f19/pa5-student}

\section*{Requirements}

This assignment is much more open-ended than previous ones. You will
implement a compiler for a language with heap allocation.\footnote{This requires
having a mostly-working compiler from past PAs, but you don't have to have a
fully-functional multi-argument calling convention working in order to
complete it, and you're free to use any code from lecture to start from.}

\subsection*{Required Features}

Your compiler must support:

\begin{itemize}

\item Some mechanism for constructing heap-allocated data that groups
together an arbitrary number of values that can have different types. You
might choose named values (like dictionaries from class) or something
strictly positional (like tuples or arrays). These can have their size
specified implicitly by syntax (like tuples) or programmatically by an
expression (like arrays with a size initialization position).

For example, you might choose {\tt (tuple $e \cdots$)} for un-named fields
where the number of expressions determines the size, {\tt (array $e$)} where
the expression must evaluate to a number and specifies the size, {\tt (dict
$x\ e\ \cdots$)} as an extension of lecture to multiple fields, or some
related idea.

\item An expression that can look up values from within heap-allocated data.
This could be by name or positionally, like {\tt (index $e\ e$)} where the
second expression is an index, {\tt (index $e\ n$)} where the second position
is a constant number, or {\tt (get $e\ x$)} where the second position is a
name. These are not the only options, and there's nothing special about the
names {\tt get} and {\tt index}.

\item If a heap-allocated value is the result of a program or printed by {\tt
print}, all of its contents should be printed in some format that makes it
clear which values are part of the same heap data. For example, in the output
all the values associated with a particular location may be enclosed in
parentheses.

\item Any other features needed to express the programs listed in the section
on required tests below.

\end{itemize}

The following features are {\bf explicitly optional}, though you may enjoy or
learn from implementing some of them, and you may want to include them to
make some of the testing work easier or support particular implementation
ideas:

\begin{itemize}
\item Type checking
\item Updating elements of heap-allocated values
\item Structural equality (it's much more implementation work than physical equality)
\item Detecting when out-of-memory occurs. Your language should be able to
allocate at least a few tens of thousands of words, but doesn't need to
detect or recover from filling up memory.
\end{itemize}

\subsection*{Required Tests}

You {\bf must} write the following programs to test your compiler. You can
pick any details you want like function names and base case behavior, they
just must be recognizable as the requested data structures and algorithms,
and they must be in files with the given names.

\begin{itemize}

\item {\tt input/simple\_examples.boa} -- A program with a number of simple
examples of constructing and accessing heap-allocated data in your language.

\item {\tt input/error1.boa} -- A program with a well-formed, type-checking,
or runtime error related to heap-allocated values.

\item {\tt input/error2.boa} -- A second program with a different
well-formed, type-checking, or runtime error related to heap-allocated
values.

\item {\tt input/error3.boa} -- A third program with a different well-formed,
type-checking, or runtime error related to heap-allocated values.

\item {\tt input/points.boa} -- A program with a function that takes an x and
a y coordinate and produces a structure with those values, and a function
that takes two points and returns a new point with their x and y coordinates
added together, along with several tests that print example output from
calling these functions.

\item {\tt input/bst.boa} -- A program that illustrates how your language
enables the creation of binary search trees, and implements functions to add
an element and check if an element is in the tree. Include several tests that
print example output from calling these functions.

\item {\tt input/list.boa} -- A program that illustrates how your language
enables the creation of linked lists, and four functions: adding an element
at the beginning, adding an element at the end, getting an element at an
index, and creating a linked list of numbers that starts at 0 and goes up to
the index of an input n.

\end{itemize}

\subsection*{Handin and Report}

You will submit your implementation to {\tt pa5}, which will run any tests
you write (including the required tests that you add). There are no
autograding tests or associated points, your submission will be graded
according to the rubric below. You can submit your code to {\tt pa4-resubmit}
if you you are strictly extending Diamondback and want to see if you've
broken any old tests with your new implementation.

To {\tt pa5-written}, submit a PDF containing the following:

\begin{itemize}

\item (10\%) The concrete grammar of your language, pointing out and
describing the new concrete syntax beyond Diamondback. Graded on clarity and
completeness (it's clear what's new, everything new is there) and if it's
accurately reflected by your {\tt parse} implementation.

\item (5\%) The definition of your language's AST, highlighting new
expressions and definitions beyond Diamondback. Graded on clarity and
completeness, and if the abstract syntax is an accurate representation of the
concrete syntax.

\item (10\%) A diagram of how heap-allocated values are arranged on the heap,
including any extra words like the size of an allocated value or other
metadata. Graded on clarity and completeness, and if it matches the
implementation of heap allocation in the compiler.

\item (55\%) The required tests above (in addition to appearing in the code
you submit, they should be in the PDF). These will be partially graded on
your explanation and provided code, and partially on if your compiler
implements them according to your expectations.

  \begin{itemize}

    \item For each of the files error1-3, show the error message and explain
    in which phase your compiler and/or runtime catches the error. (5\% each)

    \item For the others, include the actual output on your compiler, the
    output you'd like them to have (if there's any difference) and any notes
    on interesting features of that output. (10\% each)
  \end{itemize}

  \item (5\%) A description of the thing you think is most interesting or
  exciting about your design and implementation in 2-3 sentences.

  \item (5\%) A description of a feature you'd like to add to your language
  next, with an outline of how you'd add it in 2-3 sentences.

  \item (5\%) Pick two other programming languages you know that support
  heap-allocated data, and describe why your language's design is more like
  one than the other.

  \item (5\%) A list of the resources you used to complete the assignment,
  including message board posts, online resources (including resources
  outside the course readings like Stack Overflow or blog posts with design
  ideas), and students or course staff discussions you had in-person. Please
  do collaborate {\bf and give credit to your collaborators}.

\end{itemize}

\end{document}