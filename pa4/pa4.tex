
\documentclass[10pt, oneside]{article}
\usepackage[letterpaper, scale=0.9, centering]{geometry}
\usepackage{graphicx}			
\usepackage{url,hyperref}
\usepackage{amssymb,amsmath,mathpartir}
\usepackage{currfile,xstring,multicol,array}
\usepackage[dvipsnames]{xcolor}
\usepackage[labelformat=empty]{caption}

\title{UCSD CSE131 F19 -- Diamondback}

\usepackage{listings}

\lstset{
  basicstyle=\ttfamily,
  mathescape,
  columns=fullflexible
}

\setlength{\parindent}{0em}
\setlength{\parskip}{1em}
\begin{document}
\maketitle 

{\bf Due Date:} 11pm Wednesday, November 6 \hspace{2em} {\bf Closed to Collaboration}

You will implement Diamondback, a language with functions and a static type
system.

Classroom: \url{FILL} \hspace{1em} Github: \url{FILL}


\section*{Syntax}

The concrete syntax and type language for Diamondback is below. We use
$\cdots$ to indicate \textit{zero or more} of the previous
element\footnote{This is different than the last assignment, but important
for avoiding some notational clutter.}. So a program $p$ is a sequence of zero
or more definitions $d$ followed by an expression $e$, and a definition has
zero or more arguments $x \texttt{:}\tau$ and one or more expressions in its
body $e \cdots$. There are \fbox{boxes} around the new pieces of concrete
syntax.

\[
\begin{array}{ll}
\begin{array}{lrl}
e & := & n \mid \texttt{true} \mid \texttt{false} \mid x \\
  & \mid  & \texttt{(let (($x\ e$) ($x\ e$) $\cdots$) $e\ e \cdots$)} \\
  & \mid  & \texttt{(if $e\ e \ e$)} \\
  & \mid  & \texttt{($op_2$ $e\ e$)} \mid \texttt{($op_1$ $e$)} \\
  & \mid  & \texttt{(while $e\ e\ e \cdots$)} \mid \texttt{(set $x\ e$)} \\
  & \mid  & \fbox{\texttt{($f\ e\cdots$)}} \\
d & := & \fbox{\texttt{(def $f$ ($x$:$\tau \cdots$):$\tau$ $e\ e \cdots$)}} \\
p & := & \fbox{$d \cdots e$} \\
op_1 & := & \texttt{add1} \mid \texttt{sub1} \mid \texttt{isNum} \mid \texttt{isBool} \mid \fbox{\texttt{print}} \\
op_2 & := & \texttt{+} \mid \texttt{-} \mid \texttt{*} \mid \texttt{<} \mid \texttt{>} \mid \texttt{==} \\
n & := & \textrm{63-bit signed number literals} \\
x,f & := & \textrm{variable and function names} \\
\end{array}
&
\begin{array}{lrl}
\tau & := & \texttt{Num} \mid \texttt{Bool} \\
\Delta & := & \{f: \tau \cdots \rightarrow \tau, \cdots\} \\
\Delta[f] & \textit{means} & \textit{look up the type of } f \textit{ in } \Delta \\
\Gamma & := & \{x:\tau,\cdots\} \\
\Gamma[x] & \textit{means} & \textit{look up the type of } x \textit{ in } \Gamma \\
(x,\tau)::\Gamma & \textit{means} & \textit{add } x \textit{ to } \Gamma \textit{ with type } \tau \\
\Delta;\Gamma \vdash e : \tau & \textit{means} & \textit{with definitions } \Delta \textit{ and env } \Gamma, e \textit{ has type } \tau \\
\Delta \vdash_d d : \checkmark & \textit{means} & \textit{with definitions } \Delta \textit{ the definition } d \textit{ type-checks} \\
\vdash_p p : \checkmark & \textit{means} & \textit{the program } p \textit{ type checks } \\
\end{array}
\end{array}
\]

There are two different {\tt def} and function application forms because
functions are allowed to have zero arguments, but the $\cdots$ notation means
one or more. You can easily represent arguments in both cases with a list
that's allowed to be empty.

An example program that computes whether a number is even or odd extremely
inefficiently and prints several examples is:

\begin{tabular}{lr}
\begin{lstlisting}
(def even (n : Num) : Bool
  (if (== n 0) true (odd (- n 1))))
(def odd (n : Num) : Bool
  (if (== n 0) false (even (- n 1))))
(def test() : Bool
  (print (even 30))
  (print (odd 30))
  (print (even 57))
  (print (odd 57)))

(test)
\end{lstlisting}
& \hspace{5em}
\begin{lstlisting}
Expected output:
true
false
false
true
true
\end{lstlisting}
\end{tabular}

\section*{Semantics}

\subsection*{Function Definitions and Applications}

The main new feature in Diamondback is function definitions $d$ and function
applications $\texttt{($f\ e\cdots$)}$. A function application $\texttt{($f\
e_a\cdots$)}$ uses the function definition with the matching name
$\texttt{(def $f$ ($x$ :$\tau \cdots$) $e \cdots$)}$, and should evaluate to
the same result as $\texttt{(let (($x\ e_b$) $\cdots$) $e \cdots$)}$, where
the argument expressions $e_a\cdots$ come from the application, the names
$x\cdots$ come from the definition, and the body expressions $e_b\cdots$ come
from the definition.\footnote{Note that replacing application expressions
with let expressions is \emph{not} a strategy that works in general in the
compiler, because the body expressions $e_b\cdots$ could contain other
function applications. This description is, however, a useful way to describe
the behavior of a function application succinctly and is a perfectly valid
way to evaluate functions ``by hand'' when we can write out all the
intermediate steps with concrete values. See the reading for more detail:
\url{https://ucsd-cse131-f19.github.io/lectures/10-22-lec8/notes.pdf}}\footnote{We
actually need to be a little bit careful here. While this rule worked fine
for the single-argument functions in the reading, here we'd really want a
version of let that doesn't include earlier bindings in later ones to avoid
clashes between names in scope and names in the function's argument list.
This is really only relevant if you're writing things out on paper, and
doesn't affect the design of a calling convention at all.}

\subsection*{The Main Expression}

Diamondback programs are expected to have a single expression after the
definition list that is the main entry point for the program. This expression
has {\tt input} bound to the user's input by default.

As an example, consider this program:

\begin{lstlisting}
(def (abs x : Num) : Num
  (if (< x 0) (* -1 x) x))
(* (abs input) 2)
\end{lstlisting}

We could think of its evaluation taking these steps (with an input of 5):

\begin{lstlisting}
$\rightarrow$ (* (abs input) 2)
$\rightarrow$ (* (abs 5) 2)
$\rightarrow$ (* (let ((x 5)) (if (< x 0) (* -1 x) x)) 2)
$\rightarrow$ (* (if (< 5 0) (* -1 5) 5) 2)
$\rightarrow$ (* (if false (* -1 5) 5) 2)
$\rightarrow$ (* 5 2)
$\rightarrow$ 10
\end{lstlisting}

\subsection*{Printing}

Diamondback also adds a new primitive, {\tt print}, that prints a value to
the console followed by a newline. Numbers should print as their
user-interpreted value, so {\tt (print 22)} should print {\tt 22}, and {\tt
(print true)} should print {\tt true}.

The entire {\tt print} expression evaluates to the same value as its argument
(which is the value that gets printed), so {\tt (print (+ 1 7))} evaluates to
{\tt 8} in addition to printing it.

\section*{Type Checking}

Diamondback has essentially the same type rules as Copperhead for
expressions, with two changes. First, all of the rules contain a definitions
environment $\Delta$ in addition to the type environment $\Gamma$. Second,
there are two new rules:

\begin{mathpar}
\inferrule*[Left=TR-Print]
{\Gamma \vdash e : \tau}
{\Delta;\Gamma \vdash \texttt{(print $e$)} : \tau}
\and
\inferrule*[Left=TR-App]
{\Delta[f] = \tau_1 \cdots \tau_n \rightarrow \tau_r \and \left(\Delta;\Gamma \vdash e_1 : \tau_1 \right) \cdots \left(\Delta;\Gamma \vdash e_n : \tau_n \right)}
{\Delta;\Gamma \vdash \texttt{($f$ $e_1 \cdots e_n$)} : \tau_r}
\end{mathpar}

TR-Print says that {\tt print} expressions' result is the same type $\tau$ as
the argument, which matches the semantics.

In English, TR-App rule says

\begin{quote}
If the function definition $f$ has argument types $\tau_1 \cdots$ and return type $\tau_r$, and the
arguments $e_1\cdots$ of an application of $f$ have matching types, then the application has type $\tau_r$.
\end{quote}

It's common to write the type of definitions as $\tau_1 \cdots \rightarrow
\tau_r$, also called an ``arrow type'', to be evocative of taking a number of
argument types and producing a result type.

The existing rules are unchanged aside from tracking $\Delta$ (which only
TR-App uses).

There are also two new rules, one for type-checking definitions, and one for
type-checking programs. They use slightly different $\vdash$ notation with
subscripts $_d$ and $_p$ to indicate that they have meaning for pieces of
syntax other than expressions. Since, unlike expressions, we don't calculate
their overall type, we simply use $\checkmark$ in the rule to indicate that
they pass all checks.

\begin{mathpar}
\inferrule*[Left=TR-Def]
{\Delta; \{x_1:\tau_1, \cdots x_n:\tau_n\} \vdash e \cdots : \tau_r}
{\Delta \vdash_d \texttt{(def $f$ ($x_1$:$\tau_1 \cdots x_n$:$\tau_n$):$\tau_r$ $e \cdots$)} : \checkmark}
\\
\inferrule*[Left=TR-Prog]
{(\Delta; \vdash_d d_1 : \checkmark)\cdots(\Delta; \vdash_d d_n : \checkmark) \and \Delta; \{{\tt input}:{\tt Num}\} \vdash e : \tau}
{\vdash_p d_1 \cdots d_n e : \checkmark}
\end{mathpar}

Where in $\vdash_p$, $\Delta$ is constructed by mapping each definition name
to an arrow type made of its argument types and return type, so 
\texttt{(def $f$ ($x$:$\tau_1 \cdots x_n$:$\tau_n$):$\tau_r$ $e \cdots$)} would 
appear in $\Delta$ as $\{f: \tau_1\cdots\tau_n \rightarrow \tau_r\}$. In English,
TR-Def says that a definition type-checks if its body has the expected return
type $\tau_r$ when type-checked in an environment with just the arguments of
the definition mapped to their declared types. A program $p$ type-checks if
all of its definitions type-check and its main expression has some type in
the environment that assumes {\tt input} has type {\tt Num} (along with also
assuming the declared definitions).

\section*{New Errors \& Miscellaneous}

Type errors should be reported with {\tt "Type mismatch"} as usual, including
type errors resulting from the new rules, including passing the wrong number
of arguments to a function. If a function application uses a function name
that isn't defined, the compiler should report an error containing {\tt
"Unbound"}.

It's allowed for variables and functions to use the same name, so there could
be a top-level definition named {\tt f} and a variable in an argument or let
named {\tt f}.

It's a well-formedness error for multiple functions to have the same name, or
for multiple arguments within the same function to have the same name. Report
these cases with an error that contains the string {\tt "Multiple functions"}
and {\tt "Multiple bindings"} respectively.

An empty function body should be reported with {\tt "Invalid"} as with other
syntax errors.

\section*{Implementation Recommendations and Details}

\subsection*{Registers Used by {\tt main}}

You may have reasons to want to use registers like {\tt rbx, rbp, rdi} or
others. The assembly generated by gcc and clang for {\tt main} may use these
registers as well, so they should be saved at the beginning of {\tt
our\_code\_starts\_here} and restored before the final {\tt ret}. You can use
{\tt push rbx} and {\tt pop rbx} to accomplish this.

\subsection*{Stack Alignment}

Some systems require that the stack pointer {\tt rsp} be aligned at a 16-byte
(2 word) boundary before making calls into library functions that use system
calls, like {\tt printf}. If you get stack alignment segmentation
faults\footnote{We saw this in class
\url{https://github.com/ucsd-cse131-f19/ucsd-cse131-f19.github.io/blob/master/lectures/10-10-lec5/compile.ml\#L106}}
you may want to make sure your calling convention always moves {\tt rsp} by
multiples of 16, which could mean leaving an extra word of space on some
calls. For example, if your calling convention uses 2 words for the old value
of {\tt rsp} and the return address, then a function call with an {\it odd}
number of arguments could end up on an 8-byte boundary. You can test this by
using {\tt print} in functions with varying numbers of arguments.

\subsection*{Moving Labels into Memory}

In class, we used code like {\tt mov [rsp-16], after\_call} to move a label
into memory. This actually requires two instructions on some platforms. If
you see an error like {\tt "format does not support 32-bit absolute
addresses."} you may be running into this. The solution is simple, just save
the label into a register first:

\begin{lstlisting}
mov rax, after_call
mov [rsp-16], rax
\end{lstlisting}

\subsection*{Print}

While you're free to implement {\tt print} in any way you prefer that works,
one that we found expedient is to call a function defined in {\tt main.c}.
This requires using C's calling convention. On x86-64, this means moving the
argument into register {\tt rdi}, moving {\tt rsp} to free space at the top
of the stack, and then using the {\tt call} instruction to push the current
code address to the stack and jump to the function you wrote in {\tt main}.
On return, {\tt rsp} should be moved back to its original location, and your
generated code should ensure that the printed value ends up in {\tt rax}.

Keep in mind that if you use registers other than {\tt rax}, they may be
overwritten during the use of print. Wikipedia has a reasonable, brief,
accurate summary of the callee-save vs. caller-save behavior at
\url{https://en.wikipedia.org/wiki/X86_calling_conventions\#System_V_AMD64_ABI}.
We assuming that users of Diamondback will use {\tt gcc} and {\tt clang} on
systems that use this convention, not the Microsoft convention.

\section*{Describing Your Calling Convention}

As you implement Diamondback, you will need to make a number of decisions,
not least of which is the calling convention you choose, and decisions you
make around compiling application expressions and definitions. Along with
your code, you will write a desgin document as a separate PDF describing how
your calling convention works. You should make sure to cover (in whatever
order makes sense):

\begin{enumerate}
  \item A description of your calling convention in general terms:
    \begin{enumerate}
      \item What is the caller responsible for vs the callee?
      \item Do you have to do any particularly interesting work to manage the
      stack or temporary storage?
      \item Are there improvements you can imagine making in the future?
    \end{enumerate}
  \item Pick three example programs that use functions and are interesting in
  different ways, and use them to describe your calling convention:
  \begin{enumerate}
    \item Show their source, generated assembly, and output (you can
    summarize the generated code if it's quite long)
    \item Highlight the parts of the generated assembly that make the example
    interesting, distinct, and/or especially challenging to compile
  \end{enumerate}
\end{enumerate}

Still write this even if you don't think you have everything working! In that
case, in part 2, pick at least one example that {\it doesn't} work, and note
both its expected output and its actual behavior with your compiler.

\end{document}