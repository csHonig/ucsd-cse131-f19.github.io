\documentclass[10pt, oneside]{article}
\usepackage[letterpaper, scale=0.9, centering]{geometry}
\usepackage{graphicx}			
\usepackage{url,hyperref}
\usepackage{amssymb,amsmath,mathpartir}
\usepackage{currfile,xstring,multicol,array}
\usepackage[dvipsnames]{xcolor}
\usepackage[labelformat=empty]{caption}

\title{UCSD CSE131 F19 -- PA6}

\usepackage{listings}

\lstset{
  basicstyle=\ttfamily,
  mathescape,
  columns=fullflexible
}

\setlength{\parindent}{0em}
\setlength{\parskip}{0.5em}
\begin{document}
\maketitle 

{\bf Due Date:} 11pm Wednesday, November 20 \hspace{2em} {\bf Open to Collaboration}

Starter code: Start where PA5 left off, or use the PA5 starter code, or use
your PA4 implementation, or use an implementation from lecture. (See below).

\section*{Requirements}

You have two choices in this assignment: extend PA5 with more heap allocation
features or implement some compiler optimizations. You can only hand in one
of these; if for some reason you hand in to multiple versions of the
Gradescope assignment, you {\bf must} tell us which one you want us to use
before the assignment deadline.

\section*{Extend Heap Allocation}

You can submit PA6 as a completion/extension of PA5. For this version of the
PA, you must satisfy all the same requirements as PA5, with the addition of
the following required goals:

\begin{itemize}
\item Implement both {\bf structural equality} and {\bf reference equality}
on the heap allocated data you designed.
\item Make sure that {\bf structural equality} and {\bf printing heap values}
does not result in an infinite loop or stack overflow, but instead returns a
meaningful value or reports a meaningful error when a cycle is detected. (The
simplest thing is to detect a cycle and report an error).
\item Write a new test {\tt input/structequal.boa} that demonstrates how
structural equality works in your language on non-cyclic values.
\item Write a new test {\tt input/structequal-cycle.boa} that demonstrates how
structural equality works in your language on cyclic values.
\item Write a new test {\tt input/structprint-cycle.boa} that demonstrates how
printing works in your language on cyclic values.
\end{itemize}

You will hand in your code to {\tt pa6-heap-code} and a writeup to {\tt
pa6-heap-written}.

If you submit this option, {\bf include your entire writeup for PA5, with an
added section describing the new test and how it works}. Include:

\begin{enumerate}

\item A description of your approach to handling structural equality,
including relevant snippets of C or generated assembly code.

\item A description of your approach to handling cycles, including relevant
snippets of C or generated assembly code.

\item In the test section with the description of each of the three new
tests, the actual output of the generated binary (in terms of stdout/stderr),
the output you’d like it to have (if there’s any difference) and any notes on
interesting features of that output.

\end{enumerate}



\subsection*{Grading}

If you choose this option, we will grade {\bf only your PA6 submission}, and
its grade on the parts related to PA5 will count fully for your PA5
submission, with separate credit for PA6 based on the new features.

You may want to choose this option if you felt your submission for PA5 was
something you could improve upon and you'd like to explore and finish that
effort, while being awarded credit for it. You may also simply be more
interested in the structural algorithms relating to equality, printing, and
cycles.

\subsection*{Implementation Suggestions}

Structural equality is related to the recursive printing required for PA5,
but takes two arguments and compares them. Remember that if two values have
different {\it tags}, they cannot be equal.

To detect cycles, you may find it useful to store some extra metatdata for
heap allocated values. This extra metadata can be initialized to 0, and be
used during the equality and printing algorithms to track whether a
particular heap-allocated value has already been printed or checked for
equality during the current traversal of values. Note that since a value
could be printed or checked for equality more than once, this data should be
set back to 0 after checking equality or printing.


\section*{Implement Compiler Optimizations}

You will implement several compiler optimizations.

\subsection*{Required Features}

Your compiler must support:

\begin{itemize}

\item {\tt input} as in past assignments and variable assignment (the {\tt
set} expression)

\item Three optimizations:

\begin{itemize}
  \item Constant propagation (CP) of constant variables

  \item Constant folding (CF) of arithmetic

  \item Dead code elimination (DCE) for if expressions, while loops, and let
  expressions of un-unused variables.

\end{itemize}

\item A fixpointing optimization that optimizes using all of CP, CF, and DCE

\end{itemize}

\subsection*{Required Tests}

You {\bf must} write the following programs to test your compiler.

\begin{itemize}

\item {\tt input/cp.boa} A program with at least three instances of code that
is interesting for constant propagation, including at least one instance
where a constant is {\it not} propagated because it's the value of a variable
that is assigned to with {\tt set} later on.

\item {\tt input/cf.boa} -- A program with at least three instances of code
that is interesting for constant folding and where your optimization is
applied.

\item {\tt input/dce.boa} -- A program with at least three instances of code
that is interesting for dead code elimination and where your optimization is
applied.

\item {\tt input/combined.boa} -- A program that demonstrates the fixpointing
of all three optimizations where in at least two cases applying one
optimization enables another (so all three are eventually used).

\end{itemize}

\subsection*{Handin and Report}

You will submit your implementation to {\tt pa6-opt}, which will run a subset
of the PA4 grader tests to make sure your compiler hasn't broken any features
with the optimizations. Your submission will be graded according to the
rubric below. To {\tt pa6-opt-written}, submit a PDF containing the
following:

\begin{itemize}

  \item (5\%) Continuing to pass the tests from PA4.

  \item (10\%) A description of how you implemented constant propagation,
  including how you avoid propagating a constant that might later be changed
  by a variable assignment. Include snippets of OCaml code as necessary.

  \item (10\%) A description of how you implemented constant folding, including
  snippets of OCaml code as necessary. Argue in 2-3 sentences why your
  optimization will not change the {\it behavior} of any programs, 

  \item (10\%) A description of how you implemented dead code elimination,
  including snippets of OCaml code as necessary.

  \item (15\% each) For each of the required 4 tests, include the
  interestingly optimized part of the generated assembly by first running the
  compiler {\it without} your optimizations, then with them, and point out
  what changed. Write a sentence or two about the interesting aspects of the
  change.

  \item (5\%) A list of the resources you used to complete the assignment,
  including message board posts, online resources (including resources
  outside the course readings like Stack Overflow or blog posts with design
  ideas), and students or course staff discussions you had in-person. Please
  do collaborate {\bf and give credit to your collaborators}.


\end{itemize}

\subsection*{Implementation Recommendations}

As an overarching concern, you should focus on finding {\it some} cases where
you can optimize and ensure that your optimizations don't break anything.
Some particularly problematic cases are:

\begin{itemize}

\item Checking for overflow when folding arithmetic -- make sure that if you
perform constant folding on arithmetic it doesn't overflow the representable
range, and report errors as appropriate.

\item Avoiding propagating a constant past a variable assignment. For
example, these two programs are not equivalent, even though a naive constant
propagation may generate the second from the first:

\begin{lstlisting}
(let ((x 5))
  (print x)
  (set x 10)
  (print x)))
\end{lstlisting}

\begin{lstlisting}
(let ((x 5))
  (print 5)
  (set x 10)
  (print 5)))
\end{lstlisting}

We {\it don't} recommend trying (at least not at first) to make the second
{\tt print} turn into {\tt 10}.\footnote{This gets quite complex with loops
and conditionals, and gets into the topic of {\it control-flow analysis},
which is a bit beyond this assignment} Instead, it's a good idea to be {\it
safe} and not propagate the {\tt x} variable at all.

In order to detect this case, one approach is to write a function that
detects if a variable appears on the left-hand side of a {\tt set} within
another expression:

\begin{verbatim}
let rec is_variable_set_in_e (e : expr) (x : string) : bool = ...
\end{verbatim}

Alternatively, you could collect {\it all} the assignable variables in an
expression, and use that as a helper:

\begin{verbatim}
let rec get_assignable_vars (e : expr) : string list = ...
\end{verbatim}

It's up to you to design and find the places to use these functions.

\end{itemize}

\end{document}